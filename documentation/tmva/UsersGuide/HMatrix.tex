\subsection{H-Matrix discriminant}
\label{sec:hmatrix}

The origins of the H-Matrix\index{H-Matrix} approach dates back to works 
of Fisher and Mahalanobis in the context of Gaussian 
classifiers~\cite{Fisher,Mahalanobis}. 
It discriminates one class (signal) of a feature vector from another 
(background). The correlated elements of the vector are assumed to be 
Gaussian distributed, and the inverse of the covariance matrix is 
the {\em H-Matrix}. A multivariate $\chi^2$ estimator\index{Chi-squared estimator} 
is built that exploits differences in the mean values of the vector elements 
between the two classes for the purpose of discrimination.

The H-Matrix classifier as it is implemented in TMVA is equal or less performing
than the Fisher discriminant (see Sec.~\ref{sec:fisher}), and has been only  
included for completeness. 

\subsubsection{Booking options}

The H-Matrix discriminant is booked via the command:
\begin{codeexample}
\begin{tmvacode}
factory->BookMethod( Types::kHMatrix, "HMatrix", "<options>" );
\end{tmvacode}
\caption[.]{\codeexampleCaptionSize Booking of the H-Matrix classifier: the first argument is 
		   a predefined enumerator, the second argument is a user-defined 
		   string identifier, and the third argument is the configuration options string.
         Individual options are separated by a ':'. 
         See Sec.~\ref{sec:usingtmva:booking} for more information on the booking.}
\end{codeexample}

No specific options are defined for this method beyond those shared with all the other 
methods (\cf Option Table~\ref{opt:mva::methodbase} on page~\pageref{opt:mva::methodbase}).

\subsubsection{Description and implementation}

For an event $i$, each one $\chi^2$ estimator ($\chi^2_{S(B)}$) is computed for 
signal ($S$) and background ($B$), using estimates for the sample means 
($\overline x_{S(B),k}$) and covariance matrices ($C_{S(B)}$) obtained 
from the training data
\beq
  \chi^2_U(i)=\sum_{k,\ell=1}^{\Nvar}
            \left(x_k(i) - \overline x_{U,k}\right)
            C^{-1}_{U,k\ell}
            \left(x_\ell(i) - \overline x_{U,\ell}\right)\,,
\eeq
where $U=S,B$. From this, the discriminant
\beq
   \HMATRIX(i) = \frac{\chi^2_B(i)-\chi^2_S(i)}{\chi^2_B(i)+\chi^2_S(i)}\,,
\eeq
is computed to discriminate between the signal and background classes.

\subsubsection{Variable ranking}

The present implementation of the H-Matrix discriminant does not provide a ranking 
of the input variables.

\subsubsection{Performance}

The TMVA implementation of the H-Matrix classifier has been shown to underperform 
in comparison with the corresponding Fisher discriminant (\cf\  Sec.~\ref{sec:fisher}),
when using similar assumptions and complexity. It might therefore be considered to be depreciated.

