\section{Combining MVA Methods}\index{Combining}
\label{sec:combine}

In challenging classification or regression problems with a high demand for 
optimisation, or when treating feature spaces with strongly varying properties, 
it can be useful to combined MVA methods. There is large room for creativity 
inherent in such combinations. For TMVA we distinguish three classes of 
combinations:
\begin{enumerate}

\item {\em Boosting} MVA methods,\index{Boosting method}

\item {\em Categorising} MVA methods,\index{Category method}

\item Building {\em committees} of MVA methods.\index{Committee method}

\end{enumerate}
While the first two combined methods are available since TMVA~4.1, the third one is still 
under development. The MVA {\em booster} is a generalisation of the Boosted Decision 
Trees approach (see Sec.~\ref{sec:bdt})
to any method in TMVA. {\em Category methods} allow the user to specify
sub-regions of phase space, which are assigned by requirements on input 
or spectator variables, and which define disjoint sub-populations of the training 
sample, thus improving the modelling of the training data. Finally, {\em Committee methods} 
allow one to input MVA methods into other MVA methods, a procedure that can be arbitrarily 
chained. 

All of these combined methods are of course MVA methods themselves, treated just like 
any other method in TMVA for training, testing, evaluation and application. This also 
allows to categorise a committee method, for example. 

\input Boosting

\input Category
