\section{The TMVA Methods}
\label{sec:tmvaClassifiers}

All TMVA classification and regression methods (in most cases, a method serves both 
analysis goals) inherit from \code{MethodBase}, which implements basic 
functionality like the interpretation of common configuration options, the 
interaction with the training and test data sets, I/O operations and common 
performance evaluation calculus. The functionality each MVA method is required 
to implement is defined in the abstract interface \code{IMethod}.\footnote 
{ 
  Two constructors are
  implemented for each method: one that creates the method for
  a first time for training with a configuration (``option'') string
  among the arguments, and another that recreates a method from an
  existing weight file. The use of the first constructor is
  demonstrated in the example macros \code{TMVAClassification.C} and
  \code{TMVARegression.C}, while the second one is employed by the Reader in 
  \code{TMVAClassificationApplication.C} and \code{TMVARegressionApplication.C}.
  Other functions implemented by each methods are: \code{Train}
  (called for training), \code{Write/ReadWeightsToStream} (I/O of
  specific training results), \code{WriteMonitoringHistosToFile}
  (additional specific information for monitoring purposes) and 
  \code{CreateRanking} (variable ranking).  
} 
Each MVA method provides a function that creates a rank object (of 
type \code{Ranking}), which is an ordered list of the input variables 
prioritised according to criteria specific to that method. Also 
provided are brief method-specific help notes (option \code{Help}, 
switched off by default) with information on the adequate usage of 
the method and performance optimisation in case of unsatisfying 
results.

If the option \code{CreateMVAPdfs} is set TMVA creates signal and
background PDFs from the corresponding MVA response
distributions using the training sample (\cf\
Sec.~\ref{sec:usingtmva:training}). The binning and smoothing
properties of the underlying histograms can be customised via controls
implemented in the \code{PDF} class (\cf\ Sec.~\ref{sec:PDF} and Option 
Table~\ref{opt:pdf} on page~\pageref{opt:pdf}). The options specific to
\code{MethodBase} are listed in Option Table~\ref{opt:mva::methodbase}. 
They can be accessed by all MVA methods. 

The following sections describe the methods implemented in TMVA. For each method we 
proceed according to the following scheme: ($i$) a brief introduction, ($ii$) the description 
of the booking options required to configure the method, ($iii$) a description of the 
the method and TMVA implementation specifications for classification and -- where
available -- for regression, ($iv$) the properties of the 
variable ranking, and ($v$) a few comments on performance, favourable (and 
disfavoured) use cases, and comparisons with other methods.
% ======= input option table ==========================================
\begin{option}[!t]
\input optiontables/MVA__MethodBase.tex
\caption[.]{\optionCaptionSize 
   Configuration options that are common for all classifiers (but which can be controlled individually
   for each classifier). Values given are defaults. If predefined categories exist, the default category 
   is marked by a ’*’. The lower options in the table control the PDF fitting of the classifiers (required,
   \eg, for the Rarity calculation).
}
\label{opt:mva::methodbase}
\end{option}
% =====================================================================

